% GENERAL INFORMATION: HardwareX is an open access journal established to promote free and open source designing, building and customizing of scientific infrastructure (hardware). For more details on best practices for sharing open hardware see http://www.oshwa.org/sharing-best-practices/

\documentclass[11pt, letterpaper]{article}
\usepackage[utf8]{inputenc}
\usepackage[margin=1in]{geometry}
\usepackage{titlesec}
\usepackage{tabu}
\usepackage{enumitem}
\usepackage{amssymb}
\newlist{selectlist}{itemize}{2}
\setlist[selectlist]{label=$\square$,leftmargin=*,noitemsep,topsep=0pt}

\usepackage{hyperref}
\hypersetup{
    colorlinks=true,
    linkcolor=blue,
    filecolor=magenta,
    urlcolor=cyan,
}

\urlstyle{same}

% Set up the section label formatting
\titleformat{\section}[block]{\hspace{1em}\bfseries}{\thesection.}{0.5em}{}
\titleformat{\subsection}[block]{\hspace{1em}}{\thesubsection}{0.5em}{}

\begin{document}
% Create the title block
\begin{flushleft}

% Remove all text in italics when filling out the template and replace with your manuscripts corresponding text in regular font.

\setlength{\parindent}{0pt}
\setlength{\parskip}{10pt}
% \textbf{\large HardwareX article template}

%Insert title
%Max. 20 words. A good title should contain the fewest possible words that adequately describe the content of a paper.
\textbf{Title:} VentMon: and Open Source Inline Ventilator Test Fixture and Monitor

%Insert Authors
\textbf{Authors:} Robert Read, Lauria Clarke

%Insert Affiliations
\textbf{Affiliations:} Public Invention with support from the \textit{Protocol Labs} and the \textit{Mozilla Foundation}

%Insert Contact Email
%Include institutional email address of the corresponding author
\textbf{Contact email:} read.robert@gmail.com

%Insert Abstract
%Max. 200 words. Remember that the abstract is what readers see first in electronic abstracting and indexing services. This is the advertisement of your article. Make it interesting, and easy to be understood. Be accurate and specific, keep it as brief as possible.
\textbf{Abstract:} 
A device that plugs into the airway circuit and measures the parameters of an operating ventilator that are carefully controlled by clinicians.

%Insert Keywords
% At least 3 keywords. There is no limit on the no. of keywords you can list. Please remember that effective keywords should not repeat words appearing in your title, and should be neither too general nor too narrow.
\textbf{Keywords:} COVID-19, open source medical device, 

\newpage
\textbf{Specifications table:}

\tabulinesep=1ex
\begin{tabu} to \linewidth {|X|X[3,l]|}
\hline  \textbf{Hardware name} & 
  %Please specify the name of the hardware that you invented / customized
  VentMon
  \\
  \hline \textbf{Subject area} & %
  % Please state the subject area most relevant to the original community for which this hardware was developed. Example subject areas are listed below
 Educational Tools and Open Source Alternatives to Existing Infrastructure
  \\
  \hline \textbf{Hardware type} &
Field measurements and sensors,
Electrical engineering and computer science
  \\
\hline \textbf{Open source license} &
  %Please specify the open source license. For more details see the guide to authors.
 MIT License, CERN-OHL-S
  \\
\hline \textbf{Cost of hardware} &
  %Approximate cost of hardware (complete breakdown will be included in the Bill of Materials).
\$280
  \\
\hline \textbf{Source file repository} &
  % Link to the source file repository
      % insert a DOI URL to an approved source file repository:  Mendeley Data, theOSF, or Zenodo (instructions).  For example: "https://doi.org/10.5281/zenodo.3346799"
      % If there is no external repository write “Available in the article”
  \textit{DOI URL to an approved source file repository: \href{https://data.mendeley.com/}{Mendeley Data}, the \href{http://osf.io}{OSF}, or \href{https://zenodo.org/}{Zenodo} \href{https://doi.org/10.5281/zenodo.3346799}{(instructions)}. For example:} \textit{https://doi.org/10.5281/zenodo.3346799} \linebreak \linebreak
  \textit{If there is no external repository write “} \textit{Available in the article} \textit{"}
\\\hline
\end{tabu}
\end{flushleft}
% create the main body of the paper

\section{Hardware in context}
% Include a short description of the hardware, putting into context of similar open hardware and proprietary equipment in the field.
The Coronavirus pandemic has created a global shortage of ventilators. Ventilators are expensive to manufacture and during uncertain times supply chains can be disrupted increasing the cost and scarcity of these devices. Since the beginning of the pandemic in March, there has been a large movement to develop cheaper, more supply chain resilient ventilator. The urgency and goals of this movement have shifted as the global understanding of the virus has evolved. Regardless of the current state of this movement, however, the use of open source medical devices in resources limited emergency situations -- common during a global pandemic -- is a topic that has come to the forefront of may conversations about preparedness and treatment. 
Medical devices require rigorous testing before they can be used on the general public. Open source medical devices require the same level, if not more scrutiny. If an open source movement is to succeed in this field, the tools to test and validate medical devices should also be community based. The goal of VentMon is to provide an equally open source, community based verification solution to increase the efficiency and accuracy of the development process for teams making open source ventilator devices.



\section{Hardware description}
% Describe the hardware, highlighting the customization rather than the steps of the procedure. Highlight how it differs/which advantage it offers over pre-existing methods. For example, how could this hardware: be compared to other hardware in terms of cost or ease of use, be used in the development of further designs in a particular area, and so on.

% > Add 3-5 bulleted points to broadly explain to other researchers how the hardware could be potentially useful to them, for either standard or novel laboratory tasks, inside or outside of the original user community.



\begin{itemize}
\item \dots
\item \dots
\item \dots
\end{itemize}

\section{Design files}
% The  complete  design  files  must  be  either  uploaded  to  an  approved  online  repository,  uploaded  at the  time  of  submission  on  the  online  Editorial  Manager  submission  interface  as  supplementary materials [CAD files, videos,. . . ], or included in the body of the manuscript [e.g.  figures].  The three approved  online  repositories  are  Mendeley  Data,  the  Open  Science  Framework,  and  Zenodo. See repository instructions: https://doi.org/10.5281/zenodo.3346799

TODO Need to make hyperlinks colored in \textit{Location of File} column

\subsection{Design Files Summary}
% Please include a summary of all design files for your hardware by filling rows of the table below

\tabulinesep=1ex
\begin{tabu} to \linewidth {|X|X|X[1.5,1]|X[1.5,1]|}
\hline
\textbf{Design filename} & \textbf{File type} & \textbf{Open source license} & \textbf{Location of the file} \\\hline
%Insert design files
Embedded Firmware & C++ Source Code & MIT License &  \href{https://github.com/PubInv/ventmon-ventilator-inline-test-monitor/tree/master/VentMonFirmware}{VentMon Firmware}\\\hline
PIRDS Data Viewer & HTML Source Code &  MIT License &  \href{https://github.com/PubInv/vent-display}{Vent Display} \\\hline
PIRDS Data Logger & C Source Files & MIT License & \href{https://github.com/PubInv/PIRDS-logger}{PIRDS Logger} \\\hline
VentMon T0.3 PCB & PCB Design Files & MIT License & \href{https://github.com/PubInv/ventmon-ventilator-inline-test-monitor/tree/master/design/pcb}{VentMon PCB} \\\hline
Pressure Sensor and Airway Adaptor & STL File & MIT License &  \href{https://github.com/PubInv/ventmon-ventilator-inline-test-monitor/tree/master/design/3dparts}{VentMon Adaptors}\\\hline 
PIRDS Data Standard & \\\hline

% Design file 3 & File type & License & Link \\\hline

\end{tabu}

% For each design file listed in the summary above, include a short description of the file below (one or two sentences)
\textit{For each design file listed above, include a short description of the file here (one or two sentences)}

\section{Bill of materials}

% For a complex Bill of Materials, the complete Bill of Materials (editable spreadsheet file e.g., ODS file type or PDF file) can be uploaded in an open access online location such as the Open Science Frameworks repository. Include the link here. Alternatively, the Bill of Materials can be uploaded at the time of submission on the online Elsevier submission interface as supplementary material.

% > To make it easy to tell which item in the Bill of Materials corresponds to which component in your design file(s), use matching designators in both places, or otherwise explain the correspondence.

% > For material type, select from: Metal, semi-conductor, ceramic, polymer, biomaterial, organic, inorganic, composite, nanomaterial, semiconductor, non-specific, or other

{\bf Current Link:} \href{https://docs.google.com/spreadsheets/d/1EaXni1Mxnvv0jaeDQ9I1xbvPcNkRtYN8TABCBzB3cpw/edit#gid=1846296871}{VentMon HardwareX BOM}

\begin{itemize}
\item BOM needs to be converted to PDF and uploaded per instructions: \\
\textit{For a complex Bill of Materials, the complete Bill of Materials (editable spreadsheet file e.g., ODS file type or PDF file) can be uploaded in an open access online location such as the Open Science Frameworks repository. Include the link here. Alternatively, the Bill of Materials can be uploaded at the time of submission on the online Elsevier submission interface as supplementary material.}
\end{itemize}


\section{Build instructions}
%Provide detailed, step by step instructions for the construction of the reported hardware include all necessary information for reproducing the submitted hardware.
% > Explain and, when possible, characterize design decisions. Including design alternatives if they exist.
% > Use visual instructions such as schematics, images, and videos.
% > Clearly reference design files and component parts described in the Design File Summary and Bill of Materials.
% >Highlight potential safety concerns that may arise

Two versions of VentMon can be assembled depending on availability of parts and time. The most physically robust and complete version of VentMon requires the purchase and manufacture of a custom PCB as well as a number of 3D printed plastic parts. This version of the device requires the least amount of time to assemble and contains fewest discrete components. Due to the high cost of PCB manufacture and assembly, VentMon can also be created using off the shelf components readily available from DIY electronics suppliers. This version requires significantly more assembly time. Both assembly procedures are outlined below.


\subsection{PCB Based VentMon}

This version of VentMon requires two 3D printed parts  -- one encapsulated an on-board pressure sensor and one is an airway adaptor -- as well as a PCB assembly. Before beginning the assembly process make sure that you have manufactured those three parts.

\begin{enumerate}
\item
PCB Assembly

\begin{enumerate}[label=1.\arabic*]
\item Add standoffs to PCB
\begin{enumerate}[label=1.1.\arabic*]
\item take hardware and put it in the holes
\end{enumerate}

\item Mount sensor enclosure for BME280 pressure sensor
\begin{enumerate}[label=1.2.\arabic*]
\item Insert plastic part into mounting holes on PCB to check fit and alignment. The barb should face toward the outer edge of the PCB.
\item Apply glue to bottom edge and mounting pegs of plastic.
\item Insert plastic back into holes being careful not to get any glue on the sensor.
\item Allow 24 hours for glue to cure before attaching a hose to the barb.
\end{enumerate}


\end{enumerate}



\item
Enclosure

\item
Flow Sensor Assembly

\item 
Oxygen Sensor Assembly

\item
Final Assembly

\end{enumerate}

\subsection{COTS Based VentMon}
\begin{enumerate}

\item
Qwiic Shield Assembly

\item
Enclosure

\item
Flow Sensor Assembly

\item 
Oxygen Sensor Assembly

\item
Final Assembly

\end{enumerate}




\iffalse
\section{Operation instructions}
%Provide detailed instructions for the safe and proper operation of the hardware.
%> Step-by-step operational instructions for operating the hardware.
%> Use visual instructions as necessary.
%> Highlight potential safety hazards.

\textit{Provide detailed instructions for the safe and proper operation of the hardware.
\begin{itemize}
\item Step-by-step operational instructions for operating the hardware.
\item Use visual instructions as necessary.
\item Highlight potential safety hazards.
\end{itemize}}

\section{Validation and characterization}
%Demonstrate the operation of the hardware and characterize its performance over relevant critical metrics
%> Demonstrate the use of the hardware for a relevant use case.
%> If possible, characterize performance of the hardware over operational parameters.
%> Create a bulleted list that describes the capabilities (and limitations) of the hardware. For example consider descriptions of load, operation time, spin speed, coefficient of variation, accuracy, precision and etc.

\textit{Demonstrate the operation of the hardware and characterize its performance over relevant critical metrics
\begin{itemize}
\item Demonstrate the use of the hardware for a relevant use case.
\item If possible, characterize performance of the hardware over operational parameters.
\item Create a bulleted list that describes the capabilities (and limitations) of the hardware. For example consider descriptions of load, operation time, spin speed, coefficient of variation, accuracy, precision and etc.
\end{itemize}}

\section{Acknowledgements}
% [List here those individuals who provided help during the research (e.g., providing language help, writing assistance or proof reading the article, etc.).] Please also identify who provided financial support for the conduct of the research and/or preparation of the article and to briefly describe the role of the sponsor(s), if any, in study design; in the collection, analysis and interpretation of data; in the writing of the report; and in the decision to submit the article for publication. If the funding source(s) had no such involvement then this should be stated.}

\textit{[List here those individuals who provided help during the research (e.g., providing language help, writing assistance or proof reading the article, etc.).] Please also identify who provided financial support for the conduct of the research and/or preparation of the article and to briefly describe the role of the sponsor(s), if any, in study design; in the collection, analysis and interpretation of data; in the writing of the report; and in the decision to submit the article for publication. If the funding source(s) had no such involvement then this should be stated.}

\section{Declaration of interest}
% a statement must be included even if there is no conflict of interest
% All authors must disclose any financial and personal relationships with other people or organizations that could inappropriately influence (bias) their work. Examples of potential conflicts of interest include employment, consultancies, stock ownership, honoraria, paid expert testimony, patent applications/registrations, and grants or other funding. Authors must disclose any interests in a summary declaration of interest statement in the manuscript file. If there are no interests to declare then please state this: 'Declarations of interest: none'. This summary statement will be ultimately published if the article is accepted. More information.}

\textit{[a statement must be included even if there is no conflict of interest] \linebreak
All authors must disclose any financial and personal relationships with other people or organizations that could inappropriately influence (bias) their work. Examples of potential conflicts of interest include employment, consultancies, stock ownership, honoraria, paid expert testimony, patent applications/registrations, and grants or other funding. Authors must disclose any interests in a summary declaration of interest statement in the manuscript file. If there are no interests to declare then please state this: 'Declarations of interest: none'. This summary statement will be ultimately published if the article is accepted. More information.}

\section{Human and animal rights}
%> If the work involves the use of human subjects, the author should ensure that the work described has been carried out in accordance with the appropriate ethical guidelines.
%> If the work involves the use of human subjects, the author should ensure that the work described has been carried out in accordance with The Code of Ethics of the World Medical Association (Declaration of Helsinki) for experiments involving humans; Uniform Requirements for manuscripts submitted to Biomedical journals. Authors should include a statement in the manuscript that informed consent was obtained for experimentation with human subjects. The privacy rights of human subjects must always be observed.
%> All animal experiments should comply with the ARRIVE guidelines and should be carried out in accordance with the U.K. Animals (Scientific Procedures) Act, 1986 and associated guidelines, EU Directive 2010/63/EU for animal experiments, or the National Institutes of Health guide for the care and use of Laboratory animals (NIH Publications No. 8023, revised 1978) and the authors should clearly indicate in the manuscript that such guidelines have been followed

\textit{
\begin{itemize}
\item If the work involves the use of human subjects, the author should ensure that the work described has been carried out in accordance with the appropriate ethical guidelines. \item If the work involves the use of human subjects, the author should ensure that the work described has been carried out in accordance with The Code of Ethics of the World Medical Association (Declaration of Helsinki) for experiments involving humans; Uniform Requirements for manuscripts submitted to Biomedical journals. Authors should include a statement in the manuscript that informed consent was obtained for experimentation with human subjects. The privacy rights of human subjects must always be observed. \item All animal experiments should comply with the ARRIVE guidelines and should be carried out in accordance with the U.K. Animals (Scientific Procedures) Act, 1986 and associated guidelines, EU Directive 2010/63/EU for animal experiments, or the National Institutes of Health guide for the care and use of Laboratory animals (NIH Publications No. 8023, revised 1978) and the authors should clearly indicate in the manuscript that such guidelines have been followed.\end{itemize}}

\section*{References}
%> Include at least one reference, to the original publication of the hardware you customized.
%> Include other references as required. Include references to put your device in context in the literature. For more information on the reference format in HardwareX please see the Guide for Authors at: https://www.elsevier.com/journals/hardwarex/2468-0672/guide-for-authors

\textit{\begin{itemize}
\item Include at least one reference, to the original publication of the hardware you customized.
\item Include other references as required. Include references to put your device in context in the literature. For more information on the reference format in HardwareX please see the Guide for Authors at: https://www.elsevier.com/journals/hardwarex/2468-0672/guide-for-authors
\end{itemize}}

\fi

\end{document}

%> Author manuscript checklist
%> ●	HardwareX is a journal dedicated to the exhaustive and fully open source communication of advances in scientific infrastructure. Upon submission the author declares that all information necessary to reproduce the subject of the submission (e.g. bill of materials, build instructions, calibration procedures, source files, code, and safety considerations) is communicated in full and is accessible for use under an open source license.
%> ●	Is the subject of the submission under an open source license - as defined by the Open Source Hardware definition?
%> ●	Can the hardware be reproduced with the details provided in the submission?
%> ●	Are all relevant design files available on Mendeley Data, the Open Science Framework, or Zenodo repositories, described in the Summary of Design Files document, and clearly documented? (e.g. descriptive file names, commented code, labeled images, etc.)
%>      ○	If in the Open Science Framework, the repository has be registered? Instructions
%>      ○	If in Zenodo, the repository is open access and is published? Instructions
%>      ○	If in Mendeley Data, the repository is published or the sharable link was included in the additional information of the Editorial Submission interface? Instructions
%> ●	Are visual instructions used when necessary?
%> ●	Is the utility of the hardware to the scientific community?
%> ●	Is the performance of the hardware adequately demonstrated and characterized?
%> ●	Are all potential safety concerns addressed?
%> ●	For more information on the article template consult the Guide to Authors.}
